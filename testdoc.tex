%!TEX TS-program = xelatex
%!TEX encoding = UTF-8 Unicode
%\documentclass[backend=biber,style=alphabetic,citestyle=numeric]{icisfinal}
\documentclass{article}

\usepackage{graphicx}
\usepackage{framed} %for adding a framebox
\usepackage[framemethod=tikz]{mdframed} %to change the background color in 'begin{framed}
\usepackage{booktabs}
\usepackage{pbox} %for using \pbox in the big table
\usepackage{float}
\usepackage[margin=1.25in]{geometry}

\usepackage[
backend=biber,
style=apa,
% citestyle=authoryear
]{biblatex}

\def\changemargin#1#2{\list{}{\rightmargin#2\leftmargin#1}\item[]} 	%for paragraphs with...
\let\endchangemargin=\endlist

\title{Leadership in Cross Cultural Context Assignment 1 }
%\researchtype{Research in Progress}
%\shorttitle{This is my test title}
%\track{Human-computer Interaction}

\usepackage[hidelinks]{hyperref}
\addbibresource{references.bib}
% \author{Prasannjeet Singh}

\usepackage{tabularx}  % To have the table fill out the page

\begin{document}

\maketitle

%\begin{center}
%    \LARGE \textbf{Developing transport performance measures for Construction Logistic Solutions}
%\end{center}

\begin{center}
    \large \textbf{Prasannjeet Singh}\\
    \small Växjö, Sweden
\end{center}

\section{Chapter 1: Management across cultures- an introduction}
%\hline
\textbf{Key term: Multicultural competence}
    
\cite{steers2013management} define this as the ability to deal with different cultures having different assumptions, behaviors, communication style, managerial expectations in an appropriate and effective way. In short, it is the capacity to work successfully across cultures.

\textbf{Challenges and possibilities concerning multiculture wok environments}

The challenges regarding multicultural competence can be to identify the degree of multicultural competence that managers and employees possess in multicultural work environments, how to make best use of skills due to cultural difference in order to enhance corporate goals and employee welfare, how to change behaviour according to different cultures and national boundaries i.e. how quickly to get adapted to different cultures and possibility can be that multicultural competence directs social interactions and business decisions and helps businesses to avoid strategic mistakes (\cite{steers2013management}).

\textbf{Strategies to managing multicultural work teams}

According to \cite{brett2009managing} there are \textbf{four} strategies to manage multicultural work teams and that is \textbf{Adaptation} (Acknowledging cultural gaps openly and working around them), \textbf{Structural intervention} (Changing shape of the team), \textbf{Managerial intervention} (setting norms early or bringing in a higher-level manager, \textbf{Exit} (removing a team member when other options have failed).

\textit{As per these strategies, “multicultural competence” will work well with adaptation strategy because in this way cultural understanding is of utmost importance. And multicultural competence is all about having know-how of the different cultures.}


\section{Chapter 2: Global managers-challenges and responsibilities}
%\hline
\textbf{Key term: Virtual managers}
Virtual managers are the supervisors who manage people from a separate geographical location and use communication technologies rather than having face to face interaction. For meeting colleagues, they must travel to different locations (\cite{steers2013management}).

\textbf{Challenges and possibilities concerning multi-culture wok environments}

According to \cite{steers2013management} the ability to use technology to build workable networks and relationships that fulfils corporate interests and less face to face interaction can be the challenges. \cite{vinaja2003major} further explains that other challenges can be time delays in responses, communication breakdowns/noise while communication, difficulty in managing conflicts due to distance, limited available hours because of different time zones and different holidays.

\textbf{Strategies to managing multicultural work teams}

\textbf{Managerial intervention} is the best strategy for virtual managers because with this strategy norms and expectations should be set beforehand such as what should be the frequency of the reporting, meetings, updates etc (\cite{brett2009managing}). With virtual mangers, it is necessary to have multiple communication tools, schedule regular meetings, to have clear and detailed deliverables, more preference to the video calls are given rather than voice calls and emails etc.

\section{Chapter 3: Cultural environments}
%\hline
\textbf{Key term: Egalitarian Culture}

Egalitarian culture is a type of culture that favours equality of all employees. No one is considered as superior or better than other rather everyone is considered as equal in basic worth and moral status (\cite{peirce2013stanford}).

\textbf{Challenges and possibilities concerning multiculture wok environments}

Although in egalitarian company all employees share same benefits across the board, all employees share recognition of success as all employees are considered as equal contributor but in egalitarian culture there is also lack of strong leadership. This leads to more individualistic initiatives to fix issues and problems. In other words, it can be said that every worker bear responsibility and try to resolve things on its own. But in todays diverse workspace where people come from different backgrounds, employees with more hierarchical backgrounds find it difficult to adapt to egalitarianism (\cite{steers2013management}). A lack of organizational structure and authority can lead to chaos and it becomes hard to reach consensus. This leads to delay in decision making process. Sometimes, egalitarian culture finds it hard to recruit people at central positions because if candidates find another opportunity with more authority and benefits, they will switch to those jobs (\cite{steckermeier2019better}).

\textbf{Strategies to managing multicultural work teams}

For egalitarian culture, structural intervention plays an important role. Because in this way, if some team is not performing up to the mark then by changing team structure performance goals can be achieved.


\section{Chapter 4: Organizational environments}
%\hline
\textbf{Key term: Corporate culture}

Corporate culture is the beliefs and norms that dictates how interactions between employees and management should take place and how business transactions should take place both internal and external (\cite{steers2013management}).

\textbf{Challenges and possibilities concerning multiculture wok environments}

One of the challenges with corporate culture is that it makes it difficult for the employees or managers to quickly identify the type of culture (\cite{steers2013management}). Due to different cultural backgrounds, workplace interactions have become very complex leading to inefficiencies.

Corporate culture sometimes slows down company processes and make workers less productive (\cite{trefry2006double}). \cite{trefry2006double} further suggests that corporate culture of multi culture company makes team development slower because of time needed to build rapport and trust, communication among diverse people becomes time consuming because different people have different understandings of the terms and language barriers, more effort is needed to build common understanding, there are chances of more misunderstandings, conflicts, grudges etc. Overall, these aspects increase organizational costs in terms of employee dissatisfaction that leads to employee turnover and time needed to solve these issues.

\textbf{Strategies to managing multicultural work teams}

“Corporate culture” coincides with “adaption strategy” of managing multicultural work teams because it helps in understanding the gaps in the culture and make effort to solve them.

\section{Chapter 5: Communicating across cultures}
%\hline
\textbf{Key term: Language and linguistic structure}

Language plays central role to human communication. It is because of language that humans socialize, organize and manage their lives. With the help of this, humans’ express feelings and try to solve problems. Thus, language is essence to cross-cultural communication (\cite{steers2013management}). 

\textbf{Challenges and possibilities concerning multi culture wok environments}

The challenge with language in multi culture work environment is to decide which language should be made standard and official. On what basis this decision should be made. And once the main language is decided then differences come in the dialects. And the issue is that if someone does not understand the language being spoken then there is chance of misinterpretation and loss of important information. Hence according to \cite{steers2013management} it is very important to learn the language of the host country because language represents culture of the country. For culture integration, it is very important to speak the language that majority of people speak in a diverse company otherwise employee may feel alone and isolated and will get burn out and eventually decides to leave the company. 

\textbf{Strategies to managing multicultural work teams}

Language and linguistics structure come under adaptation strategy because language and culture are parts and parcels of one another. And by learning language employee can be better adapted to diverse cultures. 

\section{Chapter 6: Leading global organizations}
%\hline

\textbf{Key terms: Team leadership}

The dimensions mentioned by \cite{steers2013management} for team leadership are collaborative, integrating and diplomatic. According to them, this type of leadership is practised in Anglo, Asian and Latin America and very less in Arab regions. 

\textbf{Challenges and possibilities concerning multi culture wok environments}

Team leadership does not depend on single person who on the basis of his/her knowledge and skills accomplishes company goals rather in this type of leadership all participants are expected to have good knowledge and skill set. If team members do not have good amount of knowledge, then this leadership style will not bring any good results. In team leadership, the leader act as mediator or mentor for the whole team and every member takes responsibility of goals achievement. So, the main challenge is to have all team members responsible, skilful and knowledgeable (\cite{Cutajar2020}). 

\textbf{Strategies to managing multicultural work teams}

“Team leadership” will work well with the “structural and managerial intervention strategies” of managing multicultural work teams. Because if certain team is not performing well then either the structure of the team can be changed, or some managerial interventions can be made. 


\section{References}


% \bibliographystyle{misq}
% \bibliography{references}

\printbibliography

\end{document}
%%% Local Variables:
%%% mode: latex
%%% TeX-master: t
%%% End:
